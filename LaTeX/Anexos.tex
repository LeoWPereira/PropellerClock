% ---------- Anexos (opcionais) ----------
\chapter{Código completo do Relógio de Varredura Mecânica}

\begin{lstlisting}
/* chamada que acende o ponteiro dos minutos
 * deve ser chamada com os grupos 1, 2, 3 e 4 com o
 * seguinte comando:
 * blinkMin(g1_a, g2_a, g3_a, g4_a);
*/
void blinkMin (int g1, int g2, int g3, int g4)
{
      //acende os leds
      digitalWrite(g4,LOW);
      digitalWrite(g3,LOW);
      digitalWrite(g2,LOW);
      digitalWrite(g1,LOW);
      //mantem acesos pelo tempo de piscagem
      delay(pisca);
      //apaga os leds
      digitalWrite(g4,HIGH);
      digitalWrite(g3,HIGH);
      digitalWrite(g2,HIGH);
      digitalWrite(g1,HIGH);
}

/* chamada que acende o ponteiro da hora
 * deve ser chamada com os grupos 3 e 4, com o
 * seguinte comando:
 * blinkHora(g3_v, g4_v);
*/
void blinkHora (int g3, int g4)
{
      //acende os leds
      digitalWrite(g4,LOW);
      digitalWrite(g3,LOW);
      //mantem aceso pelo tempo de piscagem
      delay(pisca);
      //apaga os leds
      digitalWrite(g4,HIGH);
      digitalWrite(g3,HIGH);
}

/* funcao que incrementa o horario apos um minuto */
void incrementaHorario()
{
      //incrementa os minutos
      minuto++;

      //se for 60min, tem que zerar os min e incrementar as horas
      if(minuto == 60)
      {
            minuto = 0;
            hora++;

            //se a hora for igual a 12, zera
            if(hora == 12)
                hora = 0;
      }
}

/* funcao que faz tudo acontecer... faz com que os leds mostrem
 * um relogio usando as funcoes blinkHora e blinkMin

 * eh dividida em tres partes.. se o ponteiro das horas esta na
 * frente do dos minutos, se o dos minutos esta na frente ou se
 * estao alinhados. isso depende de saber qual o angulo entre os
 * ponteiro e a origem (o eixo vertical, apontando diretamente pra
 * cima).
 */
void clock ()
{
      //calcula o angulo do ponteiro em relacao a origem
      int ang_min = minuto*6;

      //calcula o angulo do ponteiro das horas
      int ang_hora = (hora*30) + (minuto/2);

      //se o ponteiro das horas estiver antes dos minutos
      if (ang_hora < ang_min)
      {
             //diferenca de tempo entre os ponteiros
             int atraso_entre = ((ang_min-ang_hora)*arco_min);

             /*tempo entre o ultimo ponteiro de uma volta e o primeiro da
             proxima*/
             int atraso_apos = ((360-ang_min+ang_hora)*arco_min);

             //delay entre a origem e o primeiro ponteiro
             int atraso_antes = hora*arco_min;

             //aguarda o tempo entre a origem e o primeiro ponteiro
             delayMicroseconds(atraso_antes);

             //loop que pisca os ponteiro sequencialmente 2400 vezes
             //variavel contadora
             int n = 0;

             for(n=0; n<rpm; n++)
             {
                   blinkHora(g3_v, g4_v);
                   delayMicroseconds(atraso_entre);
                   blinkMin(g1_a, g2_a, g3_a, g4_a);
                   delayMicroseconds(atraso_apos);
             }

             //apos um minuto (2400 voltas) chama a funcao novamente
             clock();
      }

      //se o ponteiro dos minutos esta antes das horas
      else if (ang_min<ang_hora)
      {
             //diferenca de tempo entre os ponteiros
             int atraso_entre = ((ang_hora-ang_min)*arco_min);
             /*tempo entre o ultimo ponteiro de uma volta e o primeiro da
             proxima*/
             int atraso_apos = ((360-ang_hora+ang_min)*arco_min);
             //delay entre a origem e o primeiro ponteiro
             int atraso_antes = minuto*arco_min;

             //aguarda o tempo entre a origem e o primeiro ponteiro
             delayMicroseconds(atraso_antes);

             //loop que pisca os ponteiro sequencialmente 2400 vezes
             //variavel contadora
             int n = 0;

             for(n=0; n<2400; n++)
             {
                   blinkMin(g1_a, g2_a, g3_a, g4_a);
                   delayMicroseconds(atraso_entre);
                   blinkHora(g3_v, g4_v);
                   delayMicroseconds(atraso_apos);
             }

             //apos 1min roda a funcao mais uma vez
             clock();
      }

      else
      {
            int atraso_antes = ang_min*arco_min;
            int atraso_apos = 360*arco_min;

            //loop
            int n = 0;
            for(n=0; n<2400; n++)
            {
                  blinkHora(g3_v, g4_v);
                  blinkMin(g1_a, g2_a, g3_a, g4_a);
                  delayMicroseconds(atraso_apos);
            }

            clock();
      }
}

/*
 * programa que torna um trilho de leds girando em um relogio usando o
 * fenomeno da POV.

 * as saidas pares sao verdes, as impares sao azuis e a borda eh
 * vermelha

 * o ponteiro das horas eh verde e o dos minutos eh azul
*/


int hora = 13;
int minuto = 50;

int rpm = 2400; //valor medido de rotacoes por minutos
int arco_min = 416; //valor do arco de 1min (6 graus) em micros

int accel = 0; /*valor para o cooler acelerar e posicionar a placa
na posicao norte (apontada para cima), em s*/

int pisca = 0; //valor que um led fica aceso ao piscar, em micros


//da nome para as saidas de cada grupo
int g4_v = 12;
int g4_a = 11;
int g3_v = 10;
int g3_a = 9;
int g2_v = 8;
int g2_a = 7;
int g1_v = 6;
int g1_a = 5;
int borda = 4;


void setup()
{
      //seta os pinos como saidas
      pinMode(g4_v,OUTPUT);
      pinMode(g4_a,OUTPUT);
      pinMode(g3_v,OUTPUT);
      pinMode(g3_a,OUTPUT);
      pinMode(g2_v,OUTPUT);
      pinMode(g2_a,OUTPUT);
      pinMode(g1_v,OUTPUT);
      pinMode(g1_a,OUTPUT);
      pinMode(borda,OUTPUT);

      //desliga os leds
      digitalWrite(g4_v,HIGH);
      digitalWrite(g4_a,HIGH);
      digitalWrite(g3_v,HIGH);
      digitalWrite(g3_a,HIGH);
      digitalWrite(g2_v,HIGH);
      digitalWrite(g2_a,HIGH);
      digitalWrite(g1_v,HIGH);
      digitalWrite(g1_a,HIGH);
      digitalWrite(borda,HIGH);
}

void loop ()
{
      //tempo entre ligar o arduino e colocar o cooler na tomada, em s
      delay(5000);

      //tempo para o cooler acelerar ate a posicao de origem
      delay(accel);
      //loading();
      clock();
}
\end{lstlisting} 