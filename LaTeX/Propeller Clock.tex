%% Equipe:
%% Leonardo Winter Pereira
%% Geanine Inglat
%% André Eleutério

%% Tema:
%% The Propeller Clock

\documentclass[openright]{normas-utf-tex} %openright = O capítulo começa sempre em páginas ímpares
%\documentclass[oneside]{normas-utf-tex} %oneside = Para dissertações com número de páginas menor que 100 (apenas frente da folha)

\usepackage[alf,abnt-emphasize=bf,bibjustif,recuo=0cm, abnt-etal-cite=2, abnt-etal-list=99]{abntcite} %configuração correta das referências bibliográficas.
\usepackage[brazil, portuges]{babel} % pacote português brasileiro
\usepackage[utf8]{inputenc} % pacote para acentuação direta
\usepackage{amsmath,amsfonts,amssymb} % pacote matemático
\usepackage{graphicx} % pacote gráfico
\usepackage{times} % fonte times
\usepackage{url} % fonte times
\usepackage{listings} % pacote para códigos


%\geometry{left=3.0cm,right=1.5cm,top=4cm,bottom=1cm}       Usado para realizar pequenos ajustes das margens.

% ---------- Preâmbulo ----------
\instituicao{Universidade Tecnológica Federal do Paraná} % nome da instituição
\programa{Curso em Engenharia de Computação} % nome do programa

\documento{Monografia} % [Dissertação] ou [Tese]
\nivel{Graduando} % [Mestrado] ou [Doutorado]
\titulacao{Oficinas de Integração II} % [Mestre] ou [Doutor]

\titulo{\MakeUppercase{Relógio com Varredura Mecânica}} % Título do trabalho em português
\title{\MakeUppercase{The Propeller Clock}} % Título do trabalho em inglês

\autor{Leonardo Winter Pereira, Geanine Inglat, André Eleutério} % Autor do trabalho
\cita{PEREIRA, Leonardo; INGLAT, Geanine; ELEUTÉRIO, André} % sobrenome (maiúsculas), Nome do autor do trabalho

\palavraschave{Relógio com Varredura Mecânica, Arduino, LED} % Palavras-chave do trabalho
\keywords{Propeller Clock, Arduino, LED} % Palavras-chave do trabalho em inglês

\comentario{\UTFPRdocumentodata\ apresentada ao \UTFPRprogramadata\ da \ABNTinstituicaodata\ como requisito parcial para aprovação na disciplina de "\UTFPRtitulacaodata\ ".}

\orientador{César Manuel Vargas Benitez} % nome do orientador do trabalho
\local{Curitiba} % cidade
\data{\the\year} % ano automático

%---------- Inicio do Documento ----------
\begin{document}

\capa % Geração automática da capa
\folhaderosto % Geração automática da folha de rosto

%resumo
%\begin{resumo}

%\end{resumo}

%abstract
%\begin{abstract}

%Abstract text (maximum of 500 words).

%\end{abstract}

% listas (opcionais, mas recomenda-se a partir de 5 elementos)
\listadefiguras % Geração automática da lista de figuras
\listadetabelas % Geração automática da lista de tabelas
\listadesiglas % Geração automática da lista de siglas
\listadesimbolos % Geração automática da lista de símbolos

% sumário
\sumario % Geração automática do sumário

%---------- Início do Texto ----------
%---------- Primeiro Capitulo ----------
\chapter{Introdução}
\label{chap:intro}

O propeller clock depende de um fenômeno chamado \sigla{POV}{Persistência da visão} (Persistence of vision, em inglês). Persistência da visão consiste no fenômeno observado quando um objeto visto pelo olho humano persiste na retina por uma fração de segundo após a sua percepção. Assim, imagens projetadas com uma frequência superior a 16Hz associam-se na retina sem interrupção~\cite{Wikipedia2013}~\cite{Anderson1993}.

Segundo essa teoria, ao captar uma imagem, o olho humano levaria uma fração de tempo para "apagá-la". Assim, quando os quadros de um filme de cinema são projetados na tela, o olho mistura o quadro anterior com o seguinte, provocando a ilusão de movimento: um objeto colocado à esquerda num quadro, aparecendo à direita no quadro seguinte, cria a ilusão de que o objeto se desloca da esquerda para a direita.

Posteriormente foi comprovado que este fenômeno é mais complexo, e melhor explicado quando divido entre Movimento \simbolo{$\phi$}{Phi} e Movimento \simbolo{$\beta$}{Beta}, pelo psicólogo checo Max Wertheimer. Porém esta análise fisiológica da visão foge do escopo desse trabalho~\cite{Kaufman1974}.

A persistência da visão pode ser observada no cotidiano ao se ligar um ventilador. Assim que o ventilador acelera, não é possível ver as hélices individualmente, somente um círculo.

Isso se aplica ao nosso trabalho pois ao rotacionar rapidamente uma hélice com diversos \sigla{LED}{Diodo emissor de luz}s, e ao coordenar o acendimento dos mesmos, é possível produzir imagens claramente visíveis ao olho humano.

\section{Componentes} %GEANINE
%Lista dos componentes utilizados e custo aproximado de cada um
A tabela 1 mostra uma relação dos materiais necessários para a confecção do display de varredura mecânica.

\begin{table}[!h]
	\centering
	\label{tab:lista-materiais-propeller}
	\begin{tabular}{ccc}
		\hline
		Produto & Quantidade & Valor (Individual) \\
		\hline
		Arduino Nano & 1 & R\$44,00 \\
		Latch 74LS373 & 1 & R\$1,85 \\
	    Regulador de tensão 5V - 7805 & 1 & R\$1,00 \\
		Capacitor 47\simbolo{$\mu$}{micro}F & 1 & R\$0,75 \\
        Cristal 40MHz & 1 & R\$0,70 \\
        Matriz de Resistores 330{$\Omega$} & 2 & R\$0,08 \\
        LED RGB & 16 & R\$1,10 \\
        Barra de Pinos 28 pinos & 1 & R\$0,60 \\
        Barra de Pinos 20 pinos & 2 & R\$0,60 \\
        Diodo emissor IR & 1 & R\$0,80 \\
        Fototransistor & 1 & R\$0,80 \\
        Barra de pinos 30 pinos & 1 & R\$0,60 \\
        Matriz de Contatos de Cobre & 1 & R\$8,90 \\
        Cooler 3800 RPM & 1 & R\$9,10 \\
        Slot Bateria 9V & 1 & R\$---- \\
        Total & - & R\$88.06 \\
		\hline
	\end{tabular}
    \caption[Orçamento para o Display de Varredura Mecânica]{Orçamento para o Display de Varredura Mecânica}
\end{table}


Além do desenvolvimento do display de varredura mecânica, foi necessário desenvolver um Tacômetro digital para Arduino, para ser possível testar a frequência do Cooler utilizado no projeto. Entretanto, este dispositivo será apenas utilizado como ferramenta de desenvolvimento, não fazendo parte da versão final do mesmo.

A lista de materiais para este projeto está listada na tabela 2:

\begin{table}[!h]
	\centering
	\label{tab:lista-materiais-tacometro}
	\begin{tabular}{ccc}
		\hline
		Produto & Quantidade & Valor (Individual) \\
		\hline
		Arduino Nano & 1 & R\$44,00 \\
	    Trimpot 5k\simbolo{$\Omega$}{Omega} & 1 & R\$0,58 \\
        Barra de pinos 30 pinos & 1 & R\$0,60 \\
        Transistor NPN 2n2222 & 2 & R\$0,25 \\
        LED IR & 1 & R\$0,80 \\
        Fototransistor & 1 & R\$0,80 \\
        Resistor 10{$\Omega$} & 1 & R\$0,08 \\
        Resistor 100k{$\Omega$} & 1 & R\$0,08 \\
        Resistor 15k{$\Omega$} & 1 & R\$0,08 \\
        Protoboard & 1 & R\$86,90 \\
        Total & - & R\$134,42 \\
		\hline
	\end{tabular}
    \caption[Lista de materiais utilizados no Tacômetro Digital]{Lista de materiais utilizados no Tacômetro Digital}
\end{table}

%b) "Quais partes constituem o dispositivo e como interagem entre si?" (utilizar um diagrama de blocos como auxiliar na descrição)
Uma breve descrição dos principais componentes que constituem o projeto será realizada neste momento:

\begin{itemize}
\item Arduino Nano: Será utilizado para realizar todo o processamento de informações necessário no projeto, como por exemplo, o cálculo do número de rotações da hélice, o controle dos LEDs, entre outros.
\item Latch 74LS373: Circuito integrado responsável por controlar o acendimento dos LEDs. Este componente é necessário devido ao baixo número de portas de saída por parte do Arduino.
\item Regulador de tensão 5V – 7805: O regulador de tensão é necessário para poder limitar a tensão nos LEDs, pois a diferença de potencial necessária nestes componentes é menor do que é necessário no cooler.
\item Capacitor 47 {$\mu$}F: Utilizado como filtro no regulador de tensão.
\item Cristal 40MHz: Utilizado para contar o tempo corretamente no relógio.
\item Matriz de Resistores 330{$\Omega$}: Utilizada para regular a corrente que vai para os LEDs
\item LED RGB: Utilizados para criar o efeito de ilusão de óptica ao acenderem nos momentos certos.
\item Socket 28 pinos: Utilizados para poder fixar o Arduino na matriz de contato.
\item Diodo emissor IR e Fototransistor: Utilizados para realizar a contagem de rotações por minuto do cooler.
\item Cooler 3800 RPM: Melhor escolha para o motor pois já possui um circuito pronto de potência e velocidade. Deve ser de no mínimo 3600 rpm para criar um bom efeito visual.
\end{itemize}

\section{Conteúdos Envolvidos} %LEONARDO

Para o desenvolvimento deste projeto, diversos são os conhecimentos necessários envolvendo diferentes áreas, desde eletrônica e programação até a física envolvida por trás de alguns aspectos da visão.

Na área de eletrônica, é necessário compreender os seguintes aspectos:
\begin{itemize}
\item Sensores IR: Existem sensores de infravermelho ativos e passivos. Um sensor de infravermelho ativo é composto por um emissor de luz infravermelha e um receptor, que reage a essa luz. Por sua vez, um sensor de infravermelho passivo não emite luz infravermelha, mas apenas capta esse tipo de luz no ambiente~\cite{RoboLivre2014}.

    Neste projeto, usaremos sensores de infravermelho para medir as rotações do relógio.
\item Construção da \sigla{PCI}{Placa de Circuito Impresso}: Será necessária a contrução de uma placa de circuito impresso para o projeto. Nela estarão contidos, entre outros componentes, os LEDs, posicionados da maneira correta para a visualização otimizada da saída do projeto. É necessária a compreensão de todos os passos para a criação da matriz.
\end{itemize}

Na área de programação, é fundamental compreender como programar o Arduino, pois é ele quem controla o acendimento dos LED's~\cite{Arduino2014}~\cite{ArduinoRef2014}.

Na área de mecânica é imprescindível o cálculo exato das rotações do motor. Erros, mesmo que pequenos, nesta etapa são suficientes para que a imagem final seja diferente do esperado.

Entretanto, a principal preocupação será com conceitos de POV e cadência, uma vez que abrangem a parte principal do projeto: o objetivo de causar uma ilusão aos olhos do espectador para criar uma imagem~\cite{Anderson1993}.

	O projeto também envolve conhecimento na área de eletrônica e programação, é preciso desenvolver hardware e software que apliquem corretamente os conceitos teóricos estudados. O primeiro projeto similar desenvolvido foi feito em 1995 por Bob Blick e será utilizado como referência para o desenvolvimento deste trabalho~\cite{Blick2002}.

\section{Metodologia de Execução} %TODOS JUNTOS

%a) Quais são as tarefas *específicas'* necessárias para a execução do projeto? Descrever brevemente o que envolve cada tarefa.
%b) Quais as relações de dependência entre as tarefas?
%c) Quanto tempo está previsto para cada tarefa (estimativa)?
%Sugestão: utilizar um diagrama Gantt

Será necessário construir peças, realizar testes e desenvolver o software do projeto. Construiremos um tacômetro, um motor para rotacionar a placa e uma matriz de contatos.
A construção do motor depende da construção do tacômetro, enquanto o desenvolvimento do software depende de toda a parte de hardware estar pronta pois testes são fundamentais durante o processo de desenvolvimento.

A duração de cada tarefa varia, com algumas durando apenas um dia e outras mais de uma semana. A tarefa que mais levará tempo é a redação da monografia, pois será feita com o decorrer do projeto.

O diagrama de blocos, utilizado para demonstrar de forma simples a relação entre cada componente do projeto, está mostrado na Figura 1.

\begin{figure}[!h]
	\centering
	\includegraphics{./Diagrama_Blocos.png}
	\caption[Diagrama de Blocos do projeto]{Diagrama de Blocos do projeto}
	\label{fig:Diagrama_Blocos}
\end{figure}

\begin{itemize}
\item Construção do tacômetro:
Serão usados um LED infravermelho e um receptor infravermelho, além do cooler onde serão posicionados. Com esse sistema é possível contar quantas vezes o sinal é cortado pelas hélices do cooler, e assim podemos calcular o número de rotações por minuto do sistema. Os cálculos serão efetuados com um arduino.
\item Construção da matriz de contatos:
São duas etapas, o design e a construção. No design será desenhado como a placa será impressa e onde irão os componentes. Na construção, produziremos a placa com peróxido e depois soldaremos os componentes nela.
\item Construção do motor:
Para construir o motor, as hélices do motor serão removidas e o motor será fixado em uma base com uma fonte de alimentação.
\item Integração das peças de hardware:
Para poder prosseguir com o projeto e finalizar o hardware, a matriz de contatos, o motor e o Arduino serão integrados.
\item Desenvolvimento do software:
Será desenvolvido um algoritmo que coordene o acender e apagar dos LEDs para que, ao girarem, formarão uma imagem. O arduino será programado. Para programar, usaremos uma interface no computador que nos permite programar em C e passar esse código para o arduino através da uma porta USB.
\end{itemize}

O diagrama de Gantt do projeto, com as principais tarefas especificadas, está mostrado na figura 2.

\begin{figure}[!h]
	\centering
	\includegraphics{./Diagrama_Gantt.png}
	\caption[Diagrama de Gantt do projeto]{Diagrama de Gantt do projeto}
	\label{fig:Diagrama_Gantt}
\end{figure}

\section{Banca Examinadora}
\begin{itemize}
\item Aluno convidado:
Iuri Barcat
\item Professor Orientador:
César Benitez (DAELN)
\item Professora convidada:
Leyza Dorini (DAINF)
\item Professor da disciplina:
Hugo Vieira Neto
\end{itemize} 
\input{Desenvolvimento.tex}
%---------- Quarto Capitulo ----------
\chapter{Resultados e Discussões}
\label{chap:resul}

Os resultados obtidos com o desenvolvimento deste trabalho não realizaram completamente o objetivo proposto neste projeto. As possíveis causas e soluções propostas serão comentadas mais detalhadamente na conclusão deste relatório, por enquanto nos limitaremos a relatar os resultados apresentados e discutir seus significados.

Após vários experimentos com diferentes materiais citados no desenvolvimento chegamos à um modelo final, utilizando a PCI. Com esta configuração pudemos diminuir as variáveis do projeto, trazendo mais estabilidade, porém, como veremos a seguir, não foi o suficiente.

O modelo final do projeto será apresentado na figura 12.

\begin{figure}[!h]
	\centering
	\includegraphics{./modelo_final.jpg}
	\caption[Modelo final do projeto]{Modelo final do projeto}
	\label{fig:modelo_final}
\end{figure}

Utilizando este modelo, foi possível obter um ótimo resultado estético no projeto. A placa e o motor, juntos com a bateria para alimentação dos componentes eletrônicos, encaixaram perfeitamente no hardware desenvolvido pela equipe. Porém, devido à limitações físicas, não foi possível combinar com sucesso a programação do software com o que foi construído. O algoritmo desenvolvido para o funcionamento do relógio funciona perfeitamente se as limitações do hardware forem desconsideradas, e pode ser usado em novos projetos.

Juntando as duas partes do trabalho (software e hardware) foi possível visualizar os ponteiros e a borda do relógio, entretanto, os ponteiros não ficaram fixos na posição em que deveriam estar, pois o número de rotações por minuto do motor estava variando de acordo com as condições físicas do relógio.

Realizando vários testes empíricos foi possível observar que o software funcionou de maneira correta e que a variável que não estava com o comportamento esperado era o número de rotações por minuto. A base do motor não estava completamente fixa, causando uma pequena trepidação em todo o sistema, a potência do motor era muito grande para que houvesse a estabilidade desejada. Na próxima secção deste trabalho serão apresentadas as conclusões da equipe a respeito das soluções para o problema.

\input{Conclusao.tex}

%---------- Referências ----------
\bibliography{reflatex} % Geração automática das referências a partir do arquivo reflatex.bib

\apendice
%%---------- Apêndices (opcionais) ----------
\chapter{Nome do Apêndice} 

\anexo
% ---------- Anexos (opcionais) ----------
\chapter{Código completo do Relógio de Varredura Mecânica}

\begin{lstlisting}
/* chamada que acende o ponteiro dos minutos
 * deve ser chamada com os grupos 1, 2, 3 e 4 com o
 * seguinte comando:
 * blinkMin(g1_a, g2_a, g3_a, g4_a);
*/
void blinkMin (int g1, int g2, int g3, int g4)
{
      //acende os leds
      digitalWrite(g4,LOW);
      digitalWrite(g3,LOW);
      digitalWrite(g2,LOW);
      digitalWrite(g1,LOW);
      //mantem acesos pelo tempo de piscagem
      delay(pisca);
      //apaga os leds
      digitalWrite(g4,HIGH);
      digitalWrite(g3,HIGH);
      digitalWrite(g2,HIGH);
      digitalWrite(g1,HIGH);
}

/* chamada que acende o ponteiro da hora
 * deve ser chamada com os grupos 3 e 4, com o
 * seguinte comando:
 * blinkHora(g3_v, g4_v);
*/
void blinkHora (int g3, int g4)
{
      //acende os leds
      digitalWrite(g4,LOW);
      digitalWrite(g3,LOW);
      //mantem aceso pelo tempo de piscagem
      delay(pisca);
      //apaga os leds
      digitalWrite(g4,HIGH);
      digitalWrite(g3,HIGH);
}

/* funcao que incrementa o horario apos um minuto */
void incrementaHorario()
{
      //incrementa os minutos
      minuto++;

      //se for 60min, tem que zerar os min e incrementar as horas
      if(minuto == 60)
      {
            minuto = 0;
            hora++;

            //se a hora for igual a 12, zera
            if(hora == 12)
                hora = 0;
      }
}

/* funcao que faz tudo acontecer... faz com que os leds mostrem
 * um relogio usando as funcoes blinkHora e blinkMin

 * eh dividida em tres partes.. se o ponteiro das horas esta na
 * frente do dos minutos, se o dos minutos esta na frente ou se
 * estao alinhados. isso depende de saber qual o angulo entre os
 * ponteiro e a origem (o eixo vertical, apontando diretamente pra
 * cima).
 */
void clock ()
{
      //calcula o angulo do ponteiro em relacao a origem
      int ang_min = minuto*6;

      //calcula o angulo do ponteiro das horas
      int ang_hora = (hora*30) + (minuto/2);

      //se o ponteiro das horas estiver antes dos minutos
      if (ang_hora < ang_min)
      {
             //diferenca de tempo entre os ponteiros
             int atraso_entre = ((ang_min-ang_hora)*arco_min);

             /*tempo entre o ultimo ponteiro de uma volta e o primeiro da
             proxima*/
             int atraso_apos = ((360-ang_min+ang_hora)*arco_min);

             //delay entre a origem e o primeiro ponteiro
             int atraso_antes = hora*arco_min;

             //aguarda o tempo entre a origem e o primeiro ponteiro
             delayMicroseconds(atraso_antes);

             //loop que pisca os ponteiro sequencialmente 2400 vezes
             //variavel contadora
             int n = 0;

             for(n=0; n<rpm; n++)
             {
                   blinkHora(g3_v, g4_v);
                   delayMicroseconds(atraso_entre);
                   blinkMin(g1_a, g2_a, g3_a, g4_a);
                   delayMicroseconds(atraso_apos);
             }

             //apos um minuto (2400 voltas) chama a funcao novamente
             clock();
      }

      //se o ponteiro dos minutos esta antes das horas
      else if (ang_min<ang_hora)
      {
             //diferenca de tempo entre os ponteiros
             int atraso_entre = ((ang_hora-ang_min)*arco_min);
             /*tempo entre o ultimo ponteiro de uma volta e o primeiro da
             proxima*/
             int atraso_apos = ((360-ang_hora+ang_min)*arco_min);
             //delay entre a origem e o primeiro ponteiro
             int atraso_antes = minuto*arco_min;

             //aguarda o tempo entre a origem e o primeiro ponteiro
             delayMicroseconds(atraso_antes);

             //loop que pisca os ponteiro sequencialmente 2400 vezes
             //variavel contadora
             int n = 0;

             for(n=0; n<2400; n++)
             {
                   blinkMin(g1_a, g2_a, g3_a, g4_a);
                   delayMicroseconds(atraso_entre);
                   blinkHora(g3_v, g4_v);
                   delayMicroseconds(atraso_apos);
             }

             //apos 1min roda a funcao mais uma vez
             clock();
      }

      else
      {
            int atraso_antes = ang_min*arco_min;
            int atraso_apos = 360*arco_min;

            //loop
            int n = 0;
            for(n=0; n<2400; n++)
            {
                  blinkHora(g3_v, g4_v);
                  blinkMin(g1_a, g2_a, g3_a, g4_a);
                  delayMicroseconds(atraso_apos);
            }

            clock();
      }
}

/*
 * programa que torna um trilho de leds girando em um relogio usando o
 * fenomeno da POV.

 * as saidas pares sao verdes, as impares sao azuis e a borda eh
 * vermelha

 * o ponteiro das horas eh verde e o dos minutos eh azul
*/


int hora = 13;
int minuto = 50;

int rpm = 2400; //valor medido de rotacoes por minutos
int arco_min = 416; //valor do arco de 1min (6 graus) em micros

int accel = 0; /*valor para o cooler acelerar e posicionar a placa
na posicao norte (apontada para cima), em s*/

int pisca = 0; //valor que um led fica aceso ao piscar, em micros


//da nome para as saidas de cada grupo
int g4_v = 12;
int g4_a = 11;
int g3_v = 10;
int g3_a = 9;
int g2_v = 8;
int g2_a = 7;
int g1_v = 6;
int g1_a = 5;
int borda = 4;


void setup()
{
      //seta os pinos como saidas
      pinMode(g4_v,OUTPUT);
      pinMode(g4_a,OUTPUT);
      pinMode(g3_v,OUTPUT);
      pinMode(g3_a,OUTPUT);
      pinMode(g2_v,OUTPUT);
      pinMode(g2_a,OUTPUT);
      pinMode(g1_v,OUTPUT);
      pinMode(g1_a,OUTPUT);
      pinMode(borda,OUTPUT);

      //desliga os leds
      digitalWrite(g4_v,HIGH);
      digitalWrite(g4_a,HIGH);
      digitalWrite(g3_v,HIGH);
      digitalWrite(g3_a,HIGH);
      digitalWrite(g2_v,HIGH);
      digitalWrite(g2_a,HIGH);
      digitalWrite(g1_v,HIGH);
      digitalWrite(g1_a,HIGH);
      digitalWrite(borda,HIGH);
}

void loop ()
{
      //tempo entre ligar o arduino e colocar o cooler na tomada, em s
      delay(5000);

      //tempo para o cooler acelerar ate a posicao de origem
      delay(accel);
      //loading();
      clock();
}
\end{lstlisting} 

\end{document} 